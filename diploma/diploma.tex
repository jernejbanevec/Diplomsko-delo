% POZOR! - Pazi, da imaš nastavljeno vrednost "pdfLaTeX" v zgornjem okencu
\documentclass[mat1]{article}
\usepackage[slovene]{babel}
\usepackage[utf8]{inputenc}
\usepackage[T1]{fontenc}
\usepackage{lmodern}
\usepackage{amssymb}
\usepackage{amsthm}
\usepackage{amsmath}
\usepackage{arydshln}
\usepackage{algorithm}
\usepackage{mathtools}
%\usepackage{arevmath}     % For math symbols
\usepackage[noend]{algpseudocode}
%\usepackage{fontspec,xcolor}
%\usepackage{luacode,luatexbase}
%\usepackage{xcolor}
%\usepackage{algorithm2e}
%\SetKwInput{KwInput}{Input}                % Set the Input
%\SetKwInput{KwOutput}{Output}
\newtheorem{izrek}{Izrek}
\newcommand{\norm}[1]{\left\lVert#1\right\rVert}
\newcommand\bigzero{\makebox(0,0){\text{\huge0}}}
%\renewcommand{\qedsymbol}{$\blacksquare$}

%ukazi za matematična okolja
\theoremstyle{definition}
\newtheorem{definicija}{Definicija}[section]
\newtheorem{opomba}[definicija]{Opomba}
\newtheorem{primer}{Primer}[section]

%za konec zgleda
\newcommand\xqed[1]{%
  \leavevmode\unskip\penalty9999 \hbox{}\nobreak\hfill
  \quad\hbox{#1}}
\newcommand\demo{\xqed{$\triangle$}}

%za x piši \times
% \in \mathbb{R}^{m \times n}

% Kjer so nejasnosti, dodaj znak oz. opozorilo za dopolnitev ali nejasnost
% \textbf{-- NEJASNOST!}
% \textbf{-- DOPOLNI!}

%\begin{luacode}
%local function vartosrcvar ( line )
%  return string.gsub(line, " var " , " \\src{var} ") 
%end
%luatexbase.add_to_callback( "process_input_buffer",  vartosrcvar, "var_to_srcvar")
%\end{luacode}

\begin{document}

\section{Uvod}

\subsection{Uvod}

%\textbf{-- Ta odstavek je potrebno dopolniti!}

Diskriminantna analiza se že dolga leta uporablja za določevanje lastnosti, ki poudarjajo razlike med razredi. Definirana je kot optimizacijski problem, ki vključuje kovariančne matrike, ki predstavljajo razpršenost podatkov znotraj posameznega razreda in razpršenost oziroma ločenost posameznih razredov. Diskriminantna analiza pa sama po sebi zahteva, da je ena od teh kovariančnih matrik nesingularna, kar omejuje njeno uporabo na matrikah določenih dimenzij. V nadaljevanju tako preučimo več različnih optimizacijskih kriterijev in poskušamo njihovo uporabo razširiti na vse matrike z uporabo posplošenega singularnega razcepa. Na ta način se izognemo pogoju nesingularnosti, ki ga zahteva diskriminantna analiza. Tako pridemo do posplošene diskriminantne analize, ki jo lahko uporabimo tudi, kadar je ena od kovariančnih matrik singularna (v nadaljevanju lahko vidimo, da je matrika singularna, kadar je velikost vzorca manjša, kot pa je dimenzija posamezne meritve). V delu diplomskega seminarja bomo testirali učinkovitost posplošene diskriminantne analize in jo, kjer bo to mogoče, primerjali tudi z običajno diskriminantno analizo.

\subsection{Matematični uvod}

Cilj diskriminantne analize je združiti lastnosti originalnih podatkov na način, ki kar najučinkoviteje ločuje med razredi, v katerih so podatki. Pri takšnem združevanju lastnosti podatkov se dimenzija danih podatkov zmanjša tako, da se struktura podatkov in določenih razredov kar najbolje ohrani.

Predpostavimo, da so podatki zloženi v matriko $A \in \mathbb{R}^{m \times n}$, kjer $m$ predstavlja dimenzijo posamezne meritve, $n$ pa predstavlja število meritev oz. podatkov. Denimo, da so podatki v matriki $A$ iz $k$ različnih razredov. Tako so stolpci $a_1, a_2, \ldots, a_n$ matrike $A$ združeni v $k$ podmatrik, ki predstavljajo razrede, v katerih so podatki:
$$ A = 
\begin{bmatrix}
A_1, & A_2, & \ldots, & A_k
\end{bmatrix} \text{,}
\hspace{2mm} \text{kjer} \hspace{2mm} A_i \in \mathbb{R}^{m \times n_i} \text{.}
$$ 
Tu število $n_i$ predstavlja moč indeksne množice razreda $i$. To indeksno množico razreda $i$ označimo z $N_i$. Očitno velja tudi $$
\sum_{i=1}^{k}n_i = n \text{.}$$
Matriko $A$ lahko poleg delitve na podmatrike razdelimo tudi na stolpce. Matrika $A = \left[ a_{i ,j} \right]  \in \mathbb{R}^{m \times n}$ je tako sestavljena iz $n$ posameznih stolpcev, kjer $i$-ti stolpec označimo z $a_i$:
$$ a_i =
\begin{bmatrix}
a_{1, i} \\
a_{2, i} \\
\vdots \\
a_{m, i}
\end{bmatrix}
\text{.}
$$

Cilj diskriminantne analize je najti linearno preslikavo iz $\mathbb{R}^m$ v $\mathbb{R}^\ell$, ki v novem prostoru kar najbolje poudari razrede, v katerih so podatki. Tu navadno velja $\ell \leq m - 1$, kar pomeni, da je prostor, v katerega ta linearna preslikava slika, manjdimenzionalen kot prvotni prostor. Iskano linearno preslikavo predstavimo z matriko $G^T \in \mathbb{R}^{\ell \times m}$.
Za preslikavo $G^T$ torej velja $$G^T : \mathbb{R}^m \rightarrow \mathbb{R}^\ell \text{.}$$ 
Cilj je poiskati tako preslikavo $G^T$, ki izbran $m$-dimenzionalen vektor preslika v nov vektor v $\ell$-dimezionalnem prostoru, v katerem so razredi podatkov poudarjeni, razpršenost podatkov znotraj razredov je zmanjšana, razlike med razredi pa so povečane.

Za nadaljnje izračune moramo definirati tudi centroid $i$-tega razreda, ki je izračunan kot povprečje stolpcev v $i$-tem razredu, 
$$c^{(i)} = \frac{1}{n_i} \sum_{j \in N_i} a_j \text{,}
$$
in centroid celotnih podatkov, ki je izračunan kot povprečje vseh stolpcev, to je
$$c = \frac{1}{n} \sum_{j = 1}^{n} a_j \text{.}
$$

Razpršenost podatkov znotraj posameznih razredov, razpršenost vseh podatkov ter razpršenost oziroma razlike med razredi je smiselno predstaviti s pomočjo matrik. Zato v nadaljevanju definiramo matriko
$$S_W = \sum_{i = 1}^{k} \sum_{j \in N_i}(a_j - c^{(i)})(a_j - c^{(i)})^T\text{,}$$
ki predstavlja matriko razpršenosti podatkov znotraj razredov, matriko
$$S_B = \sum_{i = 1}^{k} \sum_{j \in N_i} ( c^{(i)} - c)( c^{(i)} - c)^T = \sum_{i = 1}^{k} n_i ( c^{(i)} - c)( c^{(i)} - c)^T \text{,}$$
ki predstavlja matriko razpršenosti oziroma razlik med razredi in matriko
$$S_M = \sum_{j = 1}^{n} (a_j - c)(a_j - c)^T \text{,}$$
ki pradstavlja matriko celotne razpršenosti podatkov. Vse tri matrike so velikosti $m \times m$.
S pomočjo preslikave $G^T$ pa jih preslikamo v matrike velikosti $\ell \times \ell$ na sledeč način:
$$ S_{W}^{\ell} = G^T S_W G \text{,} \hspace{2mm} S_{B}^{\ell} = G^T S_B G \text{,} \hspace{2mm} S_{M}^{\ell} = G^T S_M G \text{.}
$$
%Med zgoraj definiranimi matrikami velja tudi enakost:
%$S_M = S_W + S_B \text{.}$ \textbf{-- DOKAŽI (pozneje, ko vpelješ še matrike H)}

Iz danih matrik razpršenosti podatkov bi radi tvorili kriterij kvalitete strukture razredov. Kriterij kvalitete strukture razredov mora imeti visoko vrednost, kadar so razredi, v katerih so podatki, strnjeni in dobro ločeni med seboj. Opazimo lahko, da $sled(S_W)$ predstavlja, kako skupaj so si podatki v posameznem razredu, saj velja
\begin{gather*} 
sled(S_W) = \sum_{t=1}^{m} \left[ \sum_{i = 1}^{k} \sum_{j \in N_i}(a_{t, j} - c_t^{(i)})^2 \right]
= \sum_{i = 1}^{k} \sum_{j \in N_i} \left[ \sum_{t=1}^{m} (a_{t, j} - c_t^{(i)})^2 \right] \\
= \sum_{i = 1}^{k} \sum_{j \in N_i} \norm{ a_j - c^{(i)}}_2^2 \text{.}
\end{gather*}
Podobno $sled(S_B)$ predstavlja ločenost med razredi, saj velja
\begin{gather*} 
sled(S_B) = \sum_{t=1}^{m} \left[ \sum_{i = 1}^{k} \sum_{j \in N_i}(c_t^{(i)} - c_t)^2\right]
= \sum_{i = 1}^{k} \sum_{j \in N_i} \left[ \sum_{t=1}^{m} (c_t^{(i)} - c_t)^2 \right] \\
= \sum_{i = 1}^{k} \sum_{j \in N_i} \norm{ c^{(i)} - c}_2^2 
= \sum_{i = 1}^{k} n_i \norm{ c^{(i)} - c}_2^2
\text{.}
\end{gather*}
Optimalna preslikava $G^T$ tako maksimizira $sled(S_{B}^{\ell})$ in minimizira $sled(S_{W}^{\ell})$. Smiselen maksimizacijski kriterij se tako zdi $$sled( G^T S_B G) / sled( G^T S_W G) \text{,}$$ ki pa ga zaradi lažjega računanja aproksimiramo kar s kriterijem $$sled((S_W^\ell)^{-1}S_B^\ell) \text{.}$$
%\textbf{-- NEJASNOST!}
% Pojasnitev prvega kriterija: Preslikana matrika S_B mora imeti čim manjšo vsoto lastnih vrednosti, preslikana matrika S_W pa čim večjo.
% Pojasnitev aproksimacijskega kriterija: Skalarni produkt lastnih vrednosti preslikanih matrik mora biti čim manjši - to je pribljižno enako

Kljub temu, da je ta optimizacijski kriterij lažje izračunljiv ima svoje pomanjkljivosti. Opazimo lahko, da kriterija ne moremo uporabiti, ko je matrika $S_W^\ell$ singularna, torej kadar je njena determinanta enaka 0. 

\begin{Large} \textbf{\#1} \end{Large}
\textbf{-- *Tu popravkov ne razumem najbolje - torej če tu dodam tisti del bo kar uredu? Ker pozneje govorim o singularnosti $S_W$ - ali naj tudi ta del izbrišem - ali naj pozneje pišem kar $S_W^\ell$?*--}

%Ker pa za determinanto matrike velja
%$$ \det(S_W^\ell) = \det(G^T S_W G) = \det(G^T) \cdot \det(S_W) \cdot \det(G) \text{,}
%$$
%je $\det(S_W^\ell)$ enaka 0 kadar je $\det(S_W)$ enaka 0, torej kadar je matrika $S_W$ singularna. 

Do te situacije pa lahko pride kar precej pogosto. Matrika $S_W \in  \mathbb{R}^{m \times m}$ je namreč singularna v vseh primerih, ko za matriko $A \in  \mathbb{R}^{m \times n}$ velja $m > n$, saj je potem po definiciji sestavljena kot vsota $n$ matrik z rangom $1$. Matrika $S_W$, ki je dimezij $m \times m$, ima tako rang manjši ali enak $n$, iz česar sledi, da je njena determinanta enaka 0.
Na primer, do tega problema pride v primeru, ko je pridobivanje podatkov drago oz. zahtevno in so pridobljeni podatki visokih dimenzij (dimenzija posameznega podatka je večja od števila vseh pridobljenih podatkov).

Obstaja več načinov, kako aplicirati diskriminantno analizo na matriki $A \in \mathbb{R}^{m \times n}$ z $m > n$. V grobem jih ločimo na tiste, kjer dimenzijo podatkov zmanjšamo v dveh korakih, in na tiste, kjer dimenzijo podatkov zmanjšamo v enem koraku. Pri prvem načinu se faza diskriminante analize nadaljuje v fazo, v kateri zanemarimo oblike posameznih razredov. Najpopularnejša metoda za prvi del tega procesa je zmanjšanje ranga matrike s pomočjo singularnega razcepa. To je tudi glavno orodje metode imenovane metoda glavnih komponent. Kakorkoli, učinkovitost dvostopenjskih načinov se precej razlikuje glede na način zmanjšanje dimenzije v prvi fazi. V diplomskem delu se bomo osredotočili na način, ki posploši diskriminantno analizo tako, da teoretično optimalno zmanjša dimenzijo podatkov, brez da bi uvedel dodaten korak. V ta namen bomo obravnavali kriterij 
$$ sled((S_2^\ell)^{-1} S_1^\ell) \hspace{1mm} \text{,}
$$
kjer matriki $S_2$ in $S_1$ predstavljata poljubno matriko izmed $S_W$, $S_B$ in $S_M$.  Kadar je matrika $S_2$ nesingularna, klasična diskriminantna analiza predstavi svojo rešitev s pomočjo posplošenega problema lastnih vrednosti. S prestrukturiranjem problema in uporabo posplošenega singularnega razcepa, pa lahko razširimo uporabnost diskriminantne analize tudi na primer, ko je matrika $S_2$ singularna.

\section{Matematična priprava - posplošeni singularni razcep}
Originalna definicija posplošenega singularnega razcepa (Van Loan) \cite{GSVD-1} je sledeča.

% ZAČETEK PRVEGA IZREKA
\begin{izrek}[Posplošeni singularni izrek (Van Loan)]
\label{izrek:SVD} Za matriki $K_A \in \mathbb{R}^{p \times m}$ z $p \geq m$ in $K_B \in \mathbb{R}^{n \times m}$ obstajata ortogonalni matriki $U \in \mathbb{R}^{p \times p}$ in $V \in \mathbb{R}^{n \times n}$ ter nesingularna matrika $X \in \mathbb{R}^{m \times m}$, da velja 
$$ U^T K_A X = 
% !!DEFINICIJA PRVE MATRIKE!!
\begin{bmatrix}
\begin{matrix}
\alpha_1 & & \\
 & \ddots & \\
 & & \alpha_m
\end{matrix} \\ \hdashline[2pt/2pt]
0_{p-m, m}
\end{bmatrix} 
% !!KONEC DEFINICIJE PRVE MATRIKE!!
\hspace{2mm} \text{in} \hspace{2mm}
 V^T K_B X = 
\Sigma_{B_q} \text{,}
$$ kjer je $q = \min(n,m) \text{,}$
% !!DEFINICIJA DRUGE MATRIKE!!
\begin{gather*}
\Sigma_{B_q} = 
\left[
\begin{array}{c;{2pt/2pt}c}
\begin{matrix}
\beta_1 & & \\
 & \ddots & \\
 & & \beta_q
\end{matrix} & 0_{q, m - q}
 \\ \hdashline[2pt/2pt]
0_{n-q, q} & 0_{n-q, m-q}
\end{array} \right] \text{,} 
\end{gather*} %\hspace{2mm}
% !!KONEC DEFINICIJE DRUGE MATRIKE!!
$
\alpha_i \geq 0 \hspace{2mm} \text{za} \hspace{2mm}
 1 \leq i \leq m \text{,} \hspace{2mm}
  \beta_i \geq 0 \hspace{2mm} \text{za} \hspace{2mm} 1 \leq i \leq q
\hspace{2mm} \text{in} \hspace{2mm} \beta_1 \geq \beta_2 \geq \ldots \geq \beta_q
\text{.}
$
\end{izrek}
\begin{proof}
% PREVERI SPODNJE PRETVORBE
% m_a = p
% m_b = n
% k = n
% k_1 = k
% n = m

Iz matrik $K_A$ in $K_B$ tvorimo združeno $(p+n)\times m$ matriko $K = \left[\begin{array}{c} K_A \\ K_B \end{array}\right]$, za katero izračunamo singularni razcep. Iz singularnega razcepa dobimo matriki $Q \in \mathbb{R}^{(p+n) \times (p+n)}$ in matriko $Z_1 \in \mathbb{R}^{m \times m}$, tako da velja 
\begin{equation}
Q^T \left[\begin{array}{c} K_A \\ K_B \end{array}\right] Z_1 = 
\begin{bmatrix}
\begin{matrix}
\gamma_1 & & \\
 & \ddots & \\
 & & \gamma_m
\end{matrix} \\ \hdashline[2pt/2pt]
0_{p+n-m, m}
\end{bmatrix} 
\text{,}
\label{eq:1}
\end{equation}
kjer za
$
k = rang(K) \hspace{2mm} \text{velja} \hspace{2mm} 
\gamma_1 \geq \ldots \geq \gamma_k > \gamma_{k+1} = \ldots \gamma_m = 0\text{.} $

Matriko $Z_1$ razdelimo na dve matriki, matriko $Z_{11} \in \mathbb{R}^{m \times k}$, ki je sestavljena iz prvih $k$ stolpcev matrike $Z_1$ in matriko $Z_{12} \in \mathbb{R}^{m \times (m-k)}$, ki je sestavljena iz preostalih $m-k$ stolpcev matrike $Z_1$. Pišemo $$Z_1 = \left[ Z_{11} \hspace{2mm} Z_{12} \right] \text{.}$$
%, lahko vidimo, da velja
%$$ Q^T K \left[\begin{array}{c} Z_{11} | Z_{12} \end{array}\right] = 
%\begin{bmatrix}
%\begin{matrix}
%\gamma_1 & & & & & \\
% & \ddots & & & & \\
% & & \gamma_k & & & \\
% & & & \gamma_{k+1} & & \\
% & & & & \ddots & \\
% & & & & & \gamma_m 
%\end{matrix} \\ \hdashline[2pt/2pt]
%0_{p+n-m, m}
%\end{bmatrix}
%\text{.}
%$$
Po predpostavki velja $p \geq m$ in ker je očitno tudi $m \geq k$, sledi $p \geq m \geq k$. Sedaj definirajmo matriko 
$$ D := diag(\gamma_1,..., \gamma_k) \in \mathbb{R}^{k \times k} \text{.}
$$
Tako iz zgornje enačbe \textit{\eqref{eq:1}} dobimo
\begin{equation}
\begin{bmatrix}
K_A Z_{11} & K_A Z_{12} \\ 
K_B Z_{11} & K_B Z_{12}
\end{bmatrix} = Q
\begin{bmatrix}
D & 0_{k, m-k} \\ 
0_{p+n-k, k} & 0_{p+n-k, m-k} 
\end{bmatrix} \text{,} \label{eq:2}
\end{equation}
od koder sledi
$$
\begin{bmatrix}
K_A Z_{11} \\ 
K_B Z_{11}
\end{bmatrix} = Q
\begin{bmatrix}
D \\ 
0
\end{bmatrix} \text{.}
$$
V kolikor še matriko $Q$ razdelimo na podmatrike na naslednji način
$$ Q = 
\begin{bmatrix}
Q_{11} & Q_{12} \\ 
Q_{21} & Q_{22}
\end{bmatrix} \text{,}
$$ kjer je matrika $Q_{11} \in \mathbb{R}^{p \times k}$, matrika $Q_{12} \in \mathbb{R}^{p \times (p+n-k)}$, matrika $Q_{21} \in \mathbb{R}^{ n \times k}$ in matrika $Q_{22} \in \mathbb{R}^{ n \times (p+n-k)}$, ugotovimo, da je
$$Q
\begin{bmatrix}
D \\ 
0
\end{bmatrix} = 
\begin{bmatrix}
Q_{11} & Q_{12} \\ 
Q_{21} & Q_{22}
\end{bmatrix}
\begin{bmatrix}
D \\ 
0
\end{bmatrix} =
\begin{bmatrix}
Q_{11} D \\ 
Q_{21} D
\end{bmatrix} \text{.}
$$
Iz tega neposredno sledi enakost
$$
K_A Z_{11} = Q_{11} D \implies K_A Z_{11} D^{-1} = Q_{11} =: K_{A_1}  \in \mathbb{R}^{p \times k} %\text{,}
$$
in enakost
$$
K_B Z_{11} = Q_{21} D \implies K_B Z_{11} D^{-1} = Q_{21} =: K_{B_1} \in \mathbb{R}^{ n \times k} \text{.}
$$
Sedaj singularni razcep naredimo na matriki $K_{B_1}$. Za matriko $K_{B_1}$ vemo, da ima isti rang kot matrika $K_B$, saj velja, da je matrika $Z_{11}$ polnega ranga (je namreč podmatrika ortogonalne matrike $Z$) in vemo, da je matrika $D^{-1}$ polnega ranga. Označimo $r = rang(K_B) = rang(K_{B_1})$. Iz singularnega razcepa za matriko $K_{B_1}$ dobimo ortogonalni matriki $V \in \mathbb{R}^{ n \times n}$ in $Z_2 \in \mathbb{R}^{ k \times k}$, da velja
\begin{equation}
V^T K_{B_1} Z_2 = \Sigma_{B_t}
 \text{,}  \label{eq:3}
\end{equation}
kjer je $t = \min\{n, k\}$, 
$\Sigma_{B_t} = 
\left[
\begin{array}{c;{2pt/2pt}c}
\begin{matrix}
\beta_1 & & \\
 & \ddots & \\
 & & \beta_t
\end{matrix} & 0_{t, k - t}
 \\ \hdashline[2pt/2pt]
0_{n-t, t} & 0_{n-t, k-t}
\end{array} \right]$
in velja 
$ \beta_1 \geq \beta_2 \geq \ldots \geq \beta_r > \beta_{r+1} = \ldots = \beta_t = 0 \text{.}$ 
\newline
Iz enačbe \textit{\eqref{eq:2}} sledi, da je
$$
K_B Z_{12} = 0_{n, m-k} \text{.}
$$
Opazimo, da velja tudi
\begin{gather*}
V^T K_B Z_1
\begin{bmatrix}
D^{-1} Z_2 & 0\\ 
0 & I_{m-k}
\end{bmatrix} = 
V^T K_B \left[ Z_{11} \hspace{2mm} Z_{12} \right]
\begin{bmatrix}
D^{-1} Z_2 & 0\\ 
0 & I_{m-k}
\end{bmatrix} = \\
V^T K_B
\begin{bmatrix}
 Z_{11} D^{-1} Z_2 & Z_{12} 
\end{bmatrix} =
\begin{bmatrix}
V^T K_B Z_{11} D^{-1} Z_2 & V^T K_B Z_{12} 
\end{bmatrix} = \\
\begin{bmatrix}
V^T K_{B_1} Z_2 & 0_{n, m-k} 
\end{bmatrix} =
\begin{bmatrix}
\Sigma_{B_t} & 0_{n, m-k} 
\end{bmatrix} = \\
\left[
\begin{array}{c;{2pt/2pt}c}
\begin{matrix}
\beta_1 & & \\
 & \ddots & \\
 & & \beta_t
\end{matrix} & 0_{t, m - t}
 \\ \hdashline[2pt/2pt]
0_{n-t, t} & 0_{n-t, m-t}
\end{array} \right]
\text{.}
\end{gather*}
Če za $q = \min\{n,m\}$ dodatno definiramo še $\beta_{t+1} = \ldots = \beta_{q} = 0$, dobimo ravno
\begin{gather*}
V^T K_B Z_1
\begin{bmatrix}
D^{-1} Z_2 & 0\\ 
0 & I_{m-k}
\end{bmatrix} = 
\left[
\begin{array}{c;{2pt/2pt}c}
\begin{matrix}
\beta_1 & & \\
 & \ddots & \\
 & & \beta_q
\end{matrix} & 0_{q, m - q}
 \\ \hdashline[2pt/2pt]
0_{n-q, q} & 0_{n-q, m-q}
\end{array} \right]
\text{,}
\end{gather*}
kar je pa ravno matrika $\Sigma_{B_q}$ iz izreka. Matriko $X$ tako definiramo na sledeči način
$$ X := Z_1 
\begin{bmatrix}
D^{-1} Z_2 & 0\\ 
0 & I_{m-k}
\end{bmatrix}
\text{.}$$

Dokazati pa moramo še, da zgornja matrika $X$ ustreza tudi enačbi iz izreka za matriko $K_A$, torej, da obstaja tudi taka matrika $U$, da velja
\begin{gather*}
 U^T K_A X = 
\begin{bmatrix}
\begin{matrix}
\alpha_1 & & \\
 & \ddots & \\
 & & \alpha_m
\end{matrix} \\ \hdashline[2pt/2pt]
0_{p-m, m}
\end{bmatrix}
\text{.} 
\end{gather*}

Ker je matrika $Q$ ortogonalna, dodatno velja $K_{A_1}^TK_{A_1} + K_{B_1}^TK_{B_1} = I_k$, kjer je $I_k$ identična matrika dimenzije $k \times k$. To enakost lahko pokažemo tako, da razpišemo spodnjo enačbo
\begin{gather*}
Q^T Q = 
\begin{bmatrix}
Q_{11}^T & Q_{21}^T \\ 
Q_{12}^T & Q_{22}^T
\end{bmatrix}
\begin{bmatrix}
Q_{11} & Q_{12} \\ 
Q_{21} & Q_{22}
\end{bmatrix} =
\begin{bmatrix}
Q_{11}^T Q_{11} + Q_{21}^T Q_{21} & Q_{11}^T Q_{12} + Q_{21}^T Q_{22} \\ 
Q_{12}^T Q_{11} + Q_{22}^T Q_{21} & Q_{12}^T Q_{12} + Q_{22}^T Q_{22}
\end{bmatrix} \\ =
\begin{bmatrix}
K_{A_1}^TK_{A_1} + K_{B_1}^TK_{B_1} & Q_{11}^T Q_{12} + Q_{21}^T Q_{22} \\ 
Q_{12}^T Q_{11} + Q_{22}^T Q_{21} & Q_{12}^T Q_{12} + Q_{22}^T Q_{22}
\end{bmatrix}  = I =
\begin{bmatrix}
I_k & 0\\ 
0 & I_{p+n-k}
\end{bmatrix} \text{.}
\end{gather*}
Definirajmo matriko $G$, ki jo dobimo s preoblikovanjem enačbe \textit{\eqref{eq:3}}:
$$ G := K_{B_1} Z_2 = V \Sigma_{B_t} \in \mathbb{R}^{n \times k} \text{.}
$$

Za matriko $K_{A_1} Z_2$ izračunamo razširjen QR razcep, $ K_{A_1} Z_2 = U R$, kjer je $U \in \mathbb{R}^{p \times p}$ ortogonalna matrika in $R \in \mathbb{R}^{p \times k}$ zgornja trapezna matrika. Tak razcep lahko naredimo na primer z uporabo Householderjevih zrcaljenj.

%STAR DEL!!!
%naredimo Householderjeva zrcaljenja: Dobimo matriko ortogonalno matriko $U^T \in \mathbb{R}^{p \times p}$, da velja
%$$ U^T K_{A_1} Z_2 = R \text{,}
%$$
%kjer je $R \in \mathbb{R}^{p \times k}$ matrika z zgornjo trapezno obliko. Ker je matrika $U$ ortogonalna, velja $ K_{A_1} Z_2 = U R$.

Opazimo lahko, da so stolpci matrike $K_{A_1} Z_2$ medsebojno ortogonalni, saj velja
\begin{gather*}
(K_{A_1} Z_2)^T (K_{A_1} Z_2) = Z_2^T K_{A_1}^T K_{A_1} Z_2 %\\
= Z_2^T (I_{k} - K_{B_1}^T K_{B_1}) Z_2 =
\\
Z_2^T Z_2 -  Z_2^T K_{B_1}^T K_{B_1} Z_2
= I_k - G^T G = I_k - \Sigma_{B_t}^T V^T V \Sigma_{B_t} 
\\
= I_k - 
\left[
\begin{array}{c;{2pt/2pt}c}
\begin{matrix}
\beta_1^2 & & \\
 & \ddots & \\
 & & \beta_t^2
\end{matrix} & 0_{t,k - t}
 \\ \hdashline[2pt/2pt]
0_{k - t, k - t} & 0_{t, k - t}
\end{array} \right] 
%\\
= 
\left[
\begin{array}{c;{2pt/2pt}c}
\begin{matrix}
1 - \beta_1^2 & & \\
 & \ddots & \\
 & & 1 - \beta_t^2
\end{matrix} & 0_{t,k - t}
 \\ \hdashline[2pt/2pt]
0_{k - t, k - t} & I_{t, k - t}
\end{array} \right]  
\\ 
= diag(1-\beta_1^2, \ldots, 1-\beta_k^2) \text{,}
\end{gather*}
kjer smo dodatno definirali še $\beta_{t+1} = \ldots = \beta_{k} = 0$. 
Iz tega sledi, da je matrika $R$ oblike 
$$ R = 
\begin{bmatrix}
\begin{matrix}
\sqrt{1 - \beta_1^2} & & \\
 & \ddots & \\
 & & \sqrt{1 - \beta_k^2}
\end{matrix} \\ \hdashline[2pt/2pt]
0_{p-k, k}
\end{bmatrix} \text{,}
$$ saj velja 
$$
(K_{A_1} Z_2)^T (K_{A_1} Z_2) = R^T U^T U R = R^T R = diag(1-\beta_1^2, \ldots, 1-\beta_k^2) \text{.}
$$
Velja tudi
\begin{gather*}
U^T K_A X =
U^T K_A Z_1
\begin{bmatrix}
D^{-1} Z_2 & 0\\ 
0 & I_{m-k}
\end{bmatrix} = 
U^T K_A %\left[\begin{array}{c} Z_{11} | Z_{12} \end{array}\right]
\left[Z_{11} \hspace{2mm} Z_{12} \right]
\begin{bmatrix}
D^{-1} Z_2 & 0\\ 
0 & I_{m-k}
\end{bmatrix} = \\
U^T K_A
\begin{bmatrix}
 Z_{11} D^{-1} Z_2 & Z_{12} 
\end{bmatrix} =
\begin{bmatrix}
U^T K_A Z_{11} D^{-1} Z_2 & U^T K_A Z_{12} 
\end{bmatrix} = \\
\begin{bmatrix}
U^T K_{A_1} Z_2 & 0_{p, m-k} 
\end{bmatrix} =
\begin{bmatrix}
U^T U R & 0_{p, m-k}
\end{bmatrix} =
\begin{bmatrix}
R & 0_{p, m-k} 
\end{bmatrix} = \\
\left[
\begin{array}{c;{2pt/2pt}c}
\begin{matrix}
\alpha_1 & & \\
 & \ddots & \\
 & & \alpha_k
\end{matrix} & 0_{k, m-k}
 \\ \hdashline[2pt/2pt]
0_{p-k, k} & 0_{p-k, m-k}
\end{array} \right] =
\left[
\begin{array}{c}
\begin{matrix}
\alpha_1 & & \\
 & \ddots & \\
 & & \alpha_m
\end{matrix}
 \\ \hdashline[2pt/2pt]
0_{p-m, m}
\end{array} \right] 
\text{,}
\end{gather*}
kjer smo definirali $\alpha_i = \sqrt{1-\beta_i^2}$ za $i = 1, \ldots, k$ in $\alpha_{k+1} = \ldots = \alpha_m = 0$.
S tem smo pokazali, da matrika $$ X = Z_1 
\begin{bmatrix}
D^{-1} Z_2 & 0\\ 
0 & I_{m-k}
\end{bmatrix}
$$
zadošča tako razcepu matrike $K_A$ kot tudi razcepu matrike $K_B$ iz izreka, kar zaključuje dokaz.
\end{proof}

Problem tega izreka pa je, da se ga ne da uporabiti, kadar dimenzije matrike $K_A$ niso ustrezne. Zaradi tega pretirano zavezujočega pogoja se odločita C.C. Paige in M.A. Saunders \cite{GSVD-2} ta posplošeni singularni izrek še dodatno posplošiti. Tako dobimo naslednji izrek:
% !!!!!! Druga posplošitev Singularnega razcepa - IZREK !!!!!
\begin{izrek}[Posplošeni singularni razcep (Paige in Saunders)]\label{GSVD2}
\label{izrek:GSVD} Naj bosta dani matriki $K_A \in \mathbb{R}^{p \times m}$ in $K_B \in \mathbb{R}^{n \times m}$. Potem za $K = \left[\begin{array}{c} K_A \\ K_B \end{array}\right]$ in $k = rang(K)$ obstajajo ortogonalne matrike $U \in \mathbb{R}^{p \times p}$, $V \in \mathbb{R}^{n \times n}$, $W \in \mathbb{R}^{k \times k}$ in $Q \in \mathbb{R}^{m \times m}$, da velja 
\begin{equation} \label{eq4}
U^T K_A Q = \Sigma_A  \left[\begin{array}{cc} W^T R, & 0 \end{array}\right] \hspace{4mm} \text{in} \hspace{4mm} V^T K_B Q = \Sigma_B  \left[\begin{array}{cc} W^T R, & 0 \end{array}\right] \text{,}
\end{equation} kjer sta
$$\Sigma_A = \begin{bmatrix} 
I_A &  & \\
 & D_A & \\
 & & 0_A  
\end{bmatrix} \hspace{4mm} \text{in} \hspace{4mm}
\Sigma_B = \begin{bmatrix} 
0_B &  & \\
 & D_B & \\
 & & I_B  
\end{bmatrix} \text{,}$$ 
$R \in \mathbb{R}^{k \times k}$ je nesingularna matrika, matriki $I_A \in \mathbb{R}^{r \times r}$ in $I_B \in \mathbb{R}^{(k-r-s) \times (k-r-s)}$ identični matriki, kjer je 
$$r = rang(K) - rang(K_B) \hspace{4mm} \text{in} \hspace{4mm} s = rang(K_A) + rang(K_B) - rang(K) \text{.}$$
Dalje sta $0_A \in \mathbb{R}^{(p-r-s) \times (k-r-s)}$ in $0_B \in \mathbb{R}^{(n-k+r) \times r}$ ničelni matriki, ki imata lahko tudi ničelno število vrstic ali stolpcev, matriki
$D_A = diag(\alpha_{r+1},..., \alpha_{r+s})$ in $D_B = diag(\beta_{r+1},..., \beta_{r+s})$ pa sta diagonalni matriki, ki zadoščata pogoju
$$1 > \alpha_{r+1} \geq \ldots \geq \alpha_{r+s} > 0 \hspace{4mm} \text{in} \hspace{4mm} 0 < \beta_{r+1} \leq \ldots \leq \beta_{r+s} < 1$$
pri
\begin{equation} \label{alpha+beta-GSVD}
\alpha_i^2 + \beta_i^2 = 1 \hspace{3mm} \text{za} \hspace{3mm} i = r+1,\ldots, r+s
\text{.}
\end{equation}
\end{izrek}
% !!!!! Dokaz druge posplošitve sing. razcepa !!!!!!!!
\begin{proof}
%Definiramo matriko $K \in \mathbb{R}^{(p+n) \times m}$, ki je sestavljena iz matrik $K_A$ in $K_B$ kot $$K:= \begin{bmatrix} 
%K_A \\
%K_B 
%\end{bmatrix}.$$
%Z $k$ označimo rang te matrike $K$, torej $k = rang(K)$.
Izračunamo singularni razcep matrike  $K$. Tako dobimo ortogonalni matriki $P \in \mathbb{R}^{(p+n) \times (p+n)}$ in $Q \in \mathbb{R}^{m \times m}$, da velja 
\begin{equation}K = 
P
\begin{bmatrix} 
R & 0_{k, m-k} \\
0_{n+p-k, k} & 0_{n+p-k, m-k} 
\end{bmatrix} \label{eq:5}
Q^T
\text{,}
\end{equation}
kjer je $R \in \mathbb{R}^{k \times k}$ diagonalna matrika ranga $k$. Matriki $Q^T$ in $P$ razdelimo na podmatrike
$$  Q^T = 
\begin{bmatrix}
Q_1^T\\
Q_2^T
\end{bmatrix}
\hspace{4mm} \text{in} \hspace{4mm}
P = \left[P_1 \hspace{2mm} P_2 \right]
=
\begin{bmatrix} 
P_{11} & P_{12} \\
P_{21} & P_{22} 
\end{bmatrix},
$$
kjer je matrika $Q_1 \in \mathbb{R}^{m \times k}$ sestavljena iz prvih $k$ stolpcev matrike $Q$, matrika $Q_2 \in \mathbb{R}^{m \times (m-k)}$ pa iz preostalih $m-k$ stolpcev matrike $Q$, podmatrike matrike $P$ pa so sledečih dimenzij: $P_{11} \in \mathbb{R}^{p \times k}$, $P_{12} \in \mathbb{R}^{p \times (p+n-k)}$, $P_{21} \in \mathbb{R}^{n \times k}$ in $P_{22} \in \mathbb{R}^{n \times (p+n-k)}$.
%, matrika $P_1 \in \mathbb{R}^{(p+m) \times k}$ iz prvih $k$ stolpcev matrike $P$ in njena podmatrika $P_{11} \in \mathbb{C}^{p \times t}$ pa iz prvih m vrstic matrike $P_1$.

Ker je $P$ ortogonalna matrika, velja $\norm{P}_2 \leq 1$ in posledično tudi $\norm{P_{11}}_2 \leq \norm{P_{1}}_2 \leq \norm{P}_2 \leq 1$. Posledično nobena singularna vrednost matrike $P_{11}$ ni večja od $1$. %Velja po izreku iz numeričnih metod, ne vem kako se kliče

Singularni razcep podobno kot na matriki $K$ naredimo tudi na matriki $P_{11}$. Tako dobimo ortogonalni matriki $U \in \mathbb{R}^{p \times p}$ in $W \in \mathbb{R}^{k \times k}$, da velja $$ P_{11} = U \Sigma_A W^T \text{,}$$ kjer je $$\Sigma_A = 
\begin{bmatrix} 
I_r &  & \\
 & D_A & \\
 & & 0_A  
\end{bmatrix} \text{,}$$ kjer je $r$ geometrična večkratnost singularne vrednosti $1$, matrika $I_r$ identična matrika dimenzije $r \times r$, matrika $D_A =
\begin{bmatrix}
\alpha_{r+1} & & \\
 & \ddots & \\
 & & \alpha_{r+s}
\end{bmatrix}$ diagonalna matrika, kjer $r+s$ predstavlja rang matrike $P_{11}$ in $0_A \in \mathbb{R}^{(p-r-s) \times (k-r-s)}$ je ničelna matrika, ki ima lahko ničelno število vrstic ali stolpcev.
% za katere velja $1 > \alpha_{r+1} \geq \ldots \geq \alpha_{r+s} > 0.$

%% Star del - ZAČETEK --
%Matriko $P_{21} W$ množimo z ortogonalnimi transformacijami tako, da uničimo vse elemente v zgornjem delu matrike $P_{21} W$. Tako dobimo ortogonalno in simetrično matriko $V \in \mathbb{R}^{n \times n}$, da velja 
%$$ V^T P_{21} W = (\ell_{ij})_{i, j} = L \text{,}$$
%kjer je matrika $L \in \mathbb{R}^{n \times k}$ spodnjetrikotna z diagonalnimi elementi večjimi od $0$. Matrika ortogonalnih transformacij $V$ je tu lahko sestavljena kar kot produkt matrik Householderjevih zrcaljenj $\tilde{P_1}$, $\tilde{P_2}$, \ldots, $\tilde{P_k}$, ki jih dobimo tako, da začnemo elemente zgornjega dela matrike $P_{21} W$ uničevati iz desne proti levi (najprej sestavimo torej matriko $\tilde{P_k}$, nato matriko $\tilde{P_{k-1}}$, itd.) in za to uporabimo manj stabilno verzijo Householderjevih zrcaljenj, tako, da je zadoščeno pogoju, da so diagonalni elementi matrike $L$ nenegativni.
%Velja torej $$V = \tilde{P_1} \tilde{P_2} \dots \tilde{P_k} \text{,}$$
%kjer je $\tilde{P_k} = I - \frac{2}{w_k^T w_k} w_k w_k^T$, $w_k = x_k + \norm{x_k}_2 e_n$, $e_n$ tu predstavlja enotski vektor z enico na $n$-tem mestu, $x_j$ pa $j$-ti stolpec matrike $P_{21} W$.
%% Star del - KONEC --

%\textbf{-- Popravljen del ZAČETEK --}

Matriko $P_{21} W$ množimo z ortogonalnimi transformacijami tako, da uničimo vse elemente v zgornjem delu matrike $P_{21} W$. Tako dobimo ortogonalno matriko $V^T \in \mathbb{R}^{n \times n}$, da velja 
$$ V^T P_{21} W = (\ell_{ij})_{i, j} = L
\text{,}$$
kjer je matrika $L \in \mathbb{R}^{n \times k}$ v primeru, ko velja $n \geq k$, spodnje trapezna matrika z nenegativnimi diagonalnimi elementi, v primeru ko velja pa $n < k$, ima obliko 
$$ L =
\left[ L_1 \hspace{2mm} L_2 \right]
\text{,} $$ kjer je $L_1$ poljubna $(k-n) \times n$ dimenzionalna matrika realnih števil, matrika $L_2 \in \mathbb{R}^{n \times n}$ pa spodnje trikotna matrika z nenegativnimi diagonalnimi elementi.
Za elemente matrike $L$ tako velja $\ell_{ij} = 0,$ ko velja $n-i > k-j$, ter $\ell_{ij} \geq 0,$ ko velja $n-i = k-j$.
Matrika ortogonalnih transformacij $V$ je tu lahko sestavljena kar kot produkt Householderjevih zrcaljenj $\tilde{P_1}$, $\tilde{P_2}$, \ldots, $\tilde{P_k}$, ki jih dobimo tako, da začnemo elemente zgornjega dela matrike $P_{21} W$ uničevati iz desne proti levi (najprej torej sestavimo matriko $\tilde{P}_k$, nato matriko $\tilde{P}_{k-1}$, itd.) in za to uporabimo manj stabilno verzijo Householderjevih zrcaljenj, tako, da je zadoščeno pogoju, da so diagonalni elementi matrike $L$ nenegativni.
Velja torej $$V^T = \tilde{P_1} \tilde{P_2} \dots \tilde{P_k} \text{.}$$
Podrobnejši opis razcepa je podan v nadaljevanju (glej \textbf{Primer 2.1.}).
%\textbf{-- Popravljen del KONEC -}

Opazimo lahko, da velja spodnja enakost
$$
\begin{bmatrix} 
U^T & 0_{m, n} \\
0_{n, m} & V^T
\end{bmatrix}
\begin{bmatrix}
P_{11} \\
P_{21}
\end{bmatrix} W = 
\begin{bmatrix}
U^T P_{11} W \\
V^TP_{21} W
\end{bmatrix} =
\begin{bmatrix}
\Sigma_A \\
L
\end{bmatrix}.
$$
Zgornja matrika $\begin{bmatrix}
\Sigma_A \\
L
\end{bmatrix}$ je ortogonalna, saj je produkt ortogonalnih matrik. 
Za matriko $L$ iz ortonormiranosti stolpcev matrike $\begin{bmatrix}
\Sigma_A \\ 
L
\end{bmatrix}$ velja, da je levih $r$ stolpcev enakih nič in tako ima matrika $L$ sledečo obliko
$$ L =
\begin{bmatrix} 
0_{n-k+r,r} & 0_{n-k+r,k-r} \\
0_{k-r, r} & L_1
\end{bmatrix} 
\text{,}
$$ kjer je $L_1 \in \mathbb{R}^{(k-r) \times (k-r)}$ spodnje trikotna matrika z nenegativnimi diagonalnimi elementi.
%\newline
Ker je desnih $k-r-s$ stolpcev matrike $\Sigma_A$ ničelnih, stolpci matrike $\begin{bmatrix}
\Sigma_A \\
L
\end{bmatrix}$ so pa po normi enaki 1, mora imeti desnih $k-r-s$ stolpcev matrike $L_1$ normo ena. Prav tako morajo biti stolpci matrike $L_1$ medsebojno pravokotni. Ker pa je matrika $L_1$ spodnje trikotna z nenegativnimi diagonalnimi elementi, je desnih $k-r-s$ diagonalnih elementov matrike $L_1$ enakih 1. To lahko bolje vidimo tako, da začnemo dopolnjevati matriko $L_1$ iz desne proti levi. Desni spodnji element matrike $L_1$ mora biti po prajšnjem enak 1 (zaradi pogoja, da imajo stolpci matrike $\begin{bmatrix}
\Sigma_A \\
L
\end{bmatrix}$ normo enako 1, ter pogoja, da so vsi ostali elementi zadnje stolpca matrike $L$ enaki 0). Posledično iz ortogonalnosti stolpcev matrike $\begin{bmatrix} \Sigma_A \\ L \end{bmatrix}$ sledi, da so vsi preostali elementi zadnje vrstice matrike $L_1$ enaki $0$, saj v kolikor bi bil katerikoli neničelen, bi kršili pogoj ortogonalnosti stolpcev. Tako nadaljujemo na naslednjem levem stolpcu in podobno pridemo do ugotovitve, da mora biti predzadnji element tega stolpca enak $1$. Enako lahko določimo vrednosti vseh $k-r-s$ desnih stolpcev matrike $L_1$. Podobno lahko določimo vrednosti tudi za $s$ levih stolpcev matrike $L_1$. V $s$-tem stolpcu matrike mora biti zaradi ortonormiranosti stolpcev $\begin{bmatrix} \Sigma_A \\ L \end{bmatrix}$ vrednost $\beta_{r+s} := \sqrt{1-\alpha^2_{r+s}}$, kjer zaradi pogoja nenegativnosti diagonalnih elementov matrike $L_1$ vzamemo pozitiven predznak. Podobno lahko pokažemo za vseh $s$ levih stolpcev matrike $L_1$. 
%
%Iz dejstva, da so stolpci matrike $\begin{bmatrix}
%\Sigma_A \\
%L
%\end{bmatrix}$ med seboj pravokotni in po normi enaki $1$, sledi, da je matrika $L_1$ diagonalna matrika, ki ima levih $s$ diagonalnih elementov med $0$ in $1$, preostalih $k-r-s$ diagonalnih elementov pa enakih $1$.
\newline
To implicira obliko matrike 
$ \begin{bmatrix}
\Sigma_A \\
L
\end{bmatrix}$, torej
$$
\begin{bmatrix}
\Sigma_A \\
L
\end{bmatrix} =
\left[
\begin{array}{c;{2pt/2pt}c;{2pt/2pt}c}
I_r & & \\ 
\hdashline[2pt/2pt]
 & D_A & \\ 
\hdashline[2pt/2pt]
 & & 0_{m-r-s, k-r-s} \\ 
\hdashline[2pt/2pt]
0_{p-k+r, r} & & \\ 
\hdashline[2pt/2pt]
 & D_B & \\ 
\hdashline[2pt/2pt]
 & & I_{k-r-s, k-r-s}
\end{array} \right] 
%=
%\begin{bmatrix}
%I_{r, r} &  & \\
% & D_A & \\
% & & 0_{m-r-s, k-r-s} \\
%0_{p-k+r, r} &  & \\
% & D_B & \\
% & & I_{k-r-s, k-r-s}
%\end{bmatrix} 
\text{,}
$$ kjer je matrika $D_B$ diagonalna matrika, 
$  D_B = 
\begin{bmatrix}
\beta_{r+1} & & \\
 & \ddots & \\
 & & \beta_{r+s} 
\end{bmatrix}\text{,}$
matrika $I_{k-r-s, k-r-s}$ identična matrika ter matrika $0_{p-k+r, r}$ ničelna matrika, ki ima lahko ničelno število vrstic ali stolpcev.
%Za matriko $L$ pa velja $L = \Sigma_B =
%\begin{bmatrix}
%0_{p-k+r, r} & & \\ 
% & D_B & \\ 
% & & I_{k-r-s, k-r-s}
%\end{bmatrix}$.
Iz ortogonalnosti matrike $\begin{bmatrix}
\Sigma_A \\
L
\end{bmatrix}$ sledi tudi, da so njeni stolpci ortonormirani, iz česar sledi $\alpha_{r+i}^2 + \beta_{r+i}^2 = 1$ za $i = 1, \ldots, s$. Matriko $L$ sedaj preimenujmo v $\Sigma_B$.

Iz enačbe  \textit{\eqref{eq:5}} in enakosti
$$
P_{11} = U \Sigma_A W^T
\hspace{2mm} \text{in} \hspace{2mm}
P_{21} = V \Sigma_B W^T
$$ dobimo
\begin{equation} \label{eq6}
\begin{split}
\begin{bmatrix} 
K_A \\
K_B 
\end{bmatrix} Q =
\begin{bmatrix} 
P_{11} & P_{12} \\
P_{21} & P_{22} 
\end{bmatrix}
\begin{bmatrix} 
R & 0_{k, m-k} \\
0_{n+p-k, k} & 0_{n+p-k, m-k}  
\end{bmatrix} =
\\
\begin{bmatrix} 
P_{11}R & 0_{p, m-k} \\
P_{21}R & 0_{n, m-k} 
\end{bmatrix} =
\begin{bmatrix} 
U \Sigma_A W^T R & 0_{p, m-k} \\
V \Sigma_B W^T R & 0_{n, m-k} 
\end{bmatrix} \text{.}
\end{split}
\end{equation}
Če začetno enačbo iz enačbe \textit{\eqref{eq6}} pomnožimo z leve z matriko 
$
\begin{bmatrix}
U^T & 0_{p, n} \\
0_{n, p} & V^T
\end{bmatrix}
$, dobimo 
$$
\begin{bmatrix}
U^T & 0_{p, n} \\
0_{n, p} & V^T
\end{bmatrix}
\begin{bmatrix} 
K_A \\
K_B 
\end{bmatrix} Q =
\begin{bmatrix}
U^T & 0_{p, n} \\
0_{n, p} & V^T
\end{bmatrix}
\begin{bmatrix} 
U \Sigma_A W^T R & 0_{p, m-k} \\
V \Sigma_B W^T R & 0_{n, m-k} 
\end{bmatrix} \text{,}
$$
iz česar sledi
%\begin{equation} \label{ena7}
$$
U^T K_A Q = 
\Sigma_A  \left[\begin{array}{cc} W^T R, & 0 \end{array}\right] \hspace{4mm} \text{in} \hspace{4mm} V^T K_B Q = \Sigma_B  \left[\begin{array}{cc} W^T R, & 0 \end{array}\right] \text{,}
%\end{equation}
$$
s čimer smo dokazali izrek.
\end{proof}

Na enostavnem primeru ponazorimo razcep, ki smo ga v zgornjem dokazu uporabili na matriki $P_{21}W$, tako da smo dobili matriko $L$ željene oblike.
\begin{primer}
Vzemimo $4 \times 2$ dimenzionalno matriko
$$ 
P_{21}W =
\begin{bmatrix}
1 & 0 \\
2 & 1 \\
2 & 1 \\
0 & 0
\end{bmatrix}
\text{.}
$$
Tu za matriko $P_{21}W \in \mathbb{R}^{n \times k}$ (v našem primeru n = 4 in k = 2) velja $n \geq k$, tako, da iščemo ortogonalno matriko $V^T \in \mathbb{R}^{n \times n}$, da velja $V^T P_{21} W = L$, kjer je $L$ spodnje trapezna matrika.

Sedaj izračunajmo manj stabilno različico Householderjevih zrcaljenj. V prvem koraku za
$$ x_2 =
\begin{bmatrix}
0 \\
1 \\
1 \\
0
\end{bmatrix}
$$ 
izračunamo 
$$ w_2 = x_2 - \norm{x_2}_2 e_n =
\begin{bmatrix}
0 \\
1 \\
1 \\
0
\end{bmatrix}
- \sqrt{2} 
\begin{bmatrix}
0 \\
0 \\
0 \\
1
\end{bmatrix} =
\begin{bmatrix}
0 \\
1 \\
1 \\
- \sqrt{2}
\end{bmatrix}
\text{.}
$$ 
Iz vektorja $w_2$ pa lahko izračunamo Householderjeva zrcaljenja
$$
P_2 = I - \frac{2}{w_2^T w_2} w_2 w_2^T = 
\begin{bmatrix}
1 & 0 & 0 & 0 \\
0 & 1 & 0 & 0 \\
0 & 0 & 1 & 0 \\
0 & 0 & 0 & 1
\end{bmatrix} - 
\frac{2}{4}
\begin{bmatrix}
0 & 0 & 0 & 0 \\
0 & 1 & 1 & -\sqrt{2} \\
0 & 1 & 1 & -\sqrt{2} \\
0 & -\sqrt{2} & -\sqrt{2} & 2
\end{bmatrix} =
\frac{1}{2}
\begin{bmatrix}
2 & 0 & 0 & 0 \\
0 & 1 & -1 & \sqrt{2} \\
0 & -1 & 1 & \sqrt{2} \\
0 & \sqrt{2} & \sqrt{2} & 0
\end{bmatrix}
\text{.}
$$
Na prvem koraku za $\tilde{P_2}$ razglasimo kar $P_2$. Izračunajmo sedaj produkt
$$
\tilde{P_2} P_{21} W =
\frac{1}{2}
\begin{bmatrix}
2 & 0 & 0 & 0 \\
0 & 1 & -1 & \sqrt{2} \\
0 & -1 & 1 & \sqrt{2} \\
0 & \sqrt{2} & \sqrt{2} & 0
\end{bmatrix}
\begin{bmatrix}
1 & 0 \\
2 & 1 \\
2 & 1 \\
0 & 0
\end{bmatrix} =
\begin{bmatrix}
2 & 0 \\
0 & 0 \\
0 & 0 \\
4\sqrt{2} & 2\sqrt{2}
\end{bmatrix} = 
\begin{bmatrix}
1 & 0 \\
0 & 0 \\
0 & 0 \\
2\sqrt{2} & \sqrt{2}
\end{bmatrix}
\text{.}
$$

Na drugem koraku vzamemo za $x_1$ naslednji levi stolpec nastalega produkta $\tilde{P_2} P_{21} W$, kjer izpustimo toliko elementov tega stolpca, kolikor je število predhodno narejenih korakov. Tako za vektor
$$ x_1 = 
\begin{bmatrix}
1\\
0\\
0\\
\end{bmatrix}
$$ 
izračunamo
$$
w_1 = x_1 - \norm{x_1} e_{n-1} = 
\begin{bmatrix}
1\\
0\\
0\\
\end{bmatrix} - 
\begin{bmatrix}
0\\
0\\
1\\
\end{bmatrix} = 
\begin{bmatrix}
1\\
0\\
-1\\
\end{bmatrix}
\text{,}
$$
kjer na $i$-tem (v tem primeru na drugem) koraku uporabimo vektor $e_{n-i+1}$ (v tem primeru torej $e_{n-1}$), ki predstavlja ničelni vektor dimenzije $n-i+1$, ki ima le na zadnjem mestu enico. S pomočjo izračunanega vektorja $w_1$ sedaj izračunajmo Householderjeva zrcaljenja
$$
P_1 = I - \frac{2}{w_1^T w_1} w_1 w_1^T = 
\begin{bmatrix}
1 & 0 & 0 \\
0 & 1 & 0 \\
0 & 0 & 1 \\
\end{bmatrix} - 
\frac{2}{2}
\begin{bmatrix}
1 & 0 & -1 \\
0 & 0 & 0 \\
-1 & 0 & 1  \\
\end{bmatrix} =
\begin{bmatrix}
0 & 0 & 1 \\
0 & 1 & 0 \\
1 & 0 & 0  \\
\end{bmatrix}
\text{.}
$$
Definiramo
$$
\tilde{P_1} = 
\begin{bmatrix}
P_1 & 0 \\
0 & I_{1,1} \\
\end{bmatrix}
 =
\begin{bmatrix}
0 & 0 & 1 & 0 \\
0 & 1 & 0 & 0 \\
1 & 0 & 0 & 0 \\
0 & 0 & 0 & 1 \\
\end{bmatrix}
\text{.}
$$
V splošnem na $i$-tem koraku za matriko $\tilde{P_i}$ vzamemo 
$
\begin{bmatrix}
P_i & 0 \\
0 & I_{i-1,i-1} \\
\end{bmatrix}
$.
Sedaj lahko poračunamo matriko 
$$
\tilde{P_1} \tilde{P_2} P_{21} W = 
\begin{bmatrix}
0 & 0 & 1 & 0 \\
0 & 1 & 0 & 0 \\
1 & 0 & 0 & 0 \\
0 & 0 & 0 & 1 \\
\end{bmatrix}
\begin{bmatrix}
1 & 0 \\
0 & 0 \\
0 & 0 \\
2\sqrt{2} & \sqrt{2}
\end{bmatrix} =
\begin{bmatrix}
0 & 0 \\
0 & 0 \\
1 & 0 \\
2\sqrt{2} & \sqrt{2}
\end{bmatrix}
$$
Opazimo, da je dobljen produkt spodnje trapezna matrika. Definiramo 
$$ V^T := \tilde{P_1} \tilde{P_2} = 
\begin{bmatrix}
0 & 0 & 1 & 0 \\
0 & 1 & 0 & 0 \\
1 & 0 & 0 & 0 \\
0 & 0 & 0 & 1 \\
\end{bmatrix}
\frac{1}{2}
\begin{bmatrix}
2 & 0 & 0 & 0 \\
0 & 1 & -1 & \sqrt{2} \\
0 & -1 & 1 & \sqrt{2} \\
0 & \sqrt{2} & \sqrt{2} & 0
\end{bmatrix} =
\begin{bmatrix}
0 & -\frac{1}{2} & \frac{1}{2} & \frac{\sqrt{2}}{2} \\
0 & \frac{1}{2} & -\frac{1}{2} & \frac{\sqrt{2}}{2} \\
1 & 0 & 0 & 0 \\
0 & \frac{\sqrt{2}}{2} & \frac{\sqrt{2}}{2} & 0
\end{bmatrix}
$$
in dobili smo ravno iskano ortogonalno matriko. Enak postopek lahko uporabimo tudi na poljubni matriki $P_{21}W \in \mathbb{R}^{n \times k}$, kjer velja $n < k$, in tako dobimo matriko $L$, ki ima obliko 
$$ L =
\left[ L_1 \hspace{2mm} L_2 \right]
\text{,} $$ kjer je $L_1$ poljubna $(k-n) \times n$ dimenzionalna matrika realnih števil, matrika $L_2 \in \mathbb{R}^{n \times n}$ pa spodnje trikotna matrika z nenegativnimi diagonalnimi elementi. \demo
\end{primer}

Iz posplošenega singularnega razcepa, ki sta ga definirala Paige in Saunders neposredno sledi Van Loanova posplošitev singularnega razcepa. 
S preoblikovanjem enačbe \textit{\eqref{eq4}}, dobimo
$$U^T K_A Q = \left[\begin{array}{cc} \Sigma_A, & 0 \end{array}\right]
\begin{bmatrix}
W^TR & 0 \\
0 & I 
\end{bmatrix} \text{.}$$
Inverz matrike
$\begin{bmatrix}
W^TR & 0 \\
0 & I 
\end{bmatrix} $
je kar matrika 
$\begin{bmatrix}
R^{-1} W & 0 \\
0 & I 
\end{bmatrix} \text{,}$
saj veljata obe enakosti iz definicije inverza, torej
$$\begin{bmatrix}
R^{-1} W & 0 \\
0 & I 
\end{bmatrix}
\begin{bmatrix}
W^TR & 0 \\
0 & I 
\end{bmatrix}
= I \hspace{3mm}\text{in} \hspace{3mm}
\begin{bmatrix}
W^TR & 0 \\
0 & I 
\end{bmatrix}
\begin{bmatrix}
R^{-1} W & 0 \\
0 & I 
\end{bmatrix}
= I \text{.}$$
Očitno sledi
$$ U^T K_A Q
\begin{bmatrix}
R^{-1} W & 0 \\
0 & I 
\end{bmatrix}
=
\left[\begin{array}{cc} \Sigma_A, & 0 \end{array}\right] \text{.}
$$

%\begin{Large} \textbf{\#3} \end{Large}
%\textbf{-- Ali mora biti ta matrika X ortogonalna? --}

Nesingularno matriko $X$ definiramo kot
\begin{equation} \label{def-X}
Q
\begin{bmatrix}
R^{-1} W & 0 \\
0 & I 
\end{bmatrix}
%\text{.}
\end{equation}
%Tako definirana matrika $X$ je ortogonalna, saj je definirana kot produkt ortogonalnih matrik.
in dobimo ravno Van Loanovo posplošitev razcepa
\begin{equation} \label{Van-Loan-Ka}
U^T K_A X = \left[\begin{array}{cc} \Sigma_A, & 0 \end{array}\right] \text{.}
\end{equation}
Podobno lahko pokažemo tudi za matriko $K_B$, za katero iz enačbe \textit{\eqref{eq4}} dobimo
\begin{equation} \label{Van-Loan-Kb}
V^T K_B X = \left[\begin{array}{cc} \Sigma_B, & 0 \end{array}\right] \text{.}
\end{equation}
Tako je definirana matrika $X$ ravno iskana matrika iz prve posplošitve singularnega razcepa in matriki $\left[\begin{array}{cc} \Sigma_A, & 0 \end{array}\right]$ ter $\left[\begin{array}{cc} \Sigma_B, & 0 \end{array}\right]$ ravno iskani diagonalni matriki.

Za nadaljnje delo definirajmo matrike\begin{equation} \label{H_W1}
H_W :=\left[A_1 - c^{(1)}e^{(1)^T}, \ldots, A_k - c^{(k)}e^{(k)^T}\right] \text{,}
\end{equation}
\begin{equation} \label{H_B1}
H_B := \left[(c^{(1)} - c)e^{(1)^T}, \ldots,(c^{(k)} - c) e^{(k)^T}\right]
\end{equation}
in
$$H_M := \left[a_1 - c, \ldots, a_n - c\right] = A - ce^T = H_W + H_B\text{,}$$
kjer velja
$e^{(i)} = (1,\ldots, 1) ^T \in \mathbb{R}^{ n_i \times 1 }$ in $e =  (1,\ldots, 1) ^T \in \mathbb{R}^{ n \times 1 }$.

S pomočjo teh matrik lahko definiramo tudi matrike razpršenosti podatkov. Matriko $S_W$ lahko definiramo kot produkt matrike $H_W$ z njeno transponiranko, torej
\begin{equation} \label{S_W-def}
S_W = H_W H_W^T 
\text{,}
\end{equation}
matriko $S_B$ lahko definiramo na podoben način kot
\begin{equation} \label{S_B-def}
S_B = H_B H_B^T 
\text{,}
\end{equation}
prav tako pa tudi matriko $S_M$
\begin{equation} \label{S_M-def}
S_M = H_M H_M^T 
\text{.}
\end{equation}

Z uporabo enačbe \textit{\eqref{S_W-def}} lahko tudi na drugačen način pokažemo prejšnjo ugotovitev, da je matrika $S_W$, kadar velja $m>n$, singularna. Razvidno je namreč, da je matrika $S_W$ definirana kot produkt matrik $H_W$ in $H_W^T$, kjer je matrika $H_W$ dimenzije $m \times n$, matrika $H_W^T$ pa dimenzije $n \times m$. V kolikor velja $m>n$, sta tako ti dve matriki največ ranga $n$.
Ker pa za rang matrike velja, da je rang produkta dveh matrik navzgor omejen z manjšim izmed rangov teh dveh posameznih matrik (torej $\text{rang}(AB) \leq \min(\text{rang}(A), \text{rang}(B)$), je posledično tudi matrika $S_W$ največ ranga $n$ in je torej (ker velja $m > n$) očitno singularna.

%\textbf{-- Tu mogoče dopiši glede matrik $K_A$ in $K_B$}

\section{Matematična rešitev problema}
V tem poglavju prikažemo uporabo posplošenega singularnega razcepa v namen razširjene uporabe posplošene diskriminantne analize.
 
\subsection{Maksimizacija optimizacijskega kriterija $J_1 = \text{sled} \left((S^\ell_2)^{-1} S^\ell_1 \right)$ za nesingularno matriko $S_2$}
Izhajamo iz maksimizacije optimizacijskega kriterija 
$$J_1(G) = \text{sled}\left((G^T S_2 G)^{-1} (G^T S_1 G)\right)$$
z izbiro optimalne preslikave $G$, kjer sta matriki $S_1$ in $S_2$ izbrani izmed matrik $S_W$, $S_B$ in $S_M$.
Matrika $S_2$ je sestavljena kot produkt matrike in transponiranke te matrike in zato je simetrično pozitivno semidefinitna (posledično so vse lastne vrednosti te matrike nenegativne). Ko pa je matrika $S_2$ nesingularna (vse lastne vrednosti matrike so različne $0$), je zato simetrično pozitivno definitna in posledično so vse lastne vrednosti te matrike pozitivne. Za simetrično pozitivno definitno matriko pa obstaja razcep Choleskega, to je obstaja spodnjetrikotna matrika $V$ s pozitivnimi elementi na diagonali, da velja
$$ S_2 = V V^T \text{.}
$$
Oglejmo si posplošeni problem lastnih vrednosti za matriki $S_2$ in $S_1$, kjer za ti dve matriki iščemo takšen $\lambda_i \in \mathbb{R}$ in takšen neničelen vektor $x_i \in \mathbb{R}^m$, da velja
\begin{equation} \label{pos-prob-last-v}
S_1 x_i = \lambda_i S_2 x_i
\text{.}
\end{equation}
Če sedaj matriko $S_2$ nadomestimo z matriko $VV^T$, ki jo dobimo iz razcepa Choleskega, dobimo
$$
S_1 x_i = \lambda_i V V^T x_i = V \lambda_i V^T x_i
\text{,}
$$
kar lahko z leve pomnožimo z $V^{-1}$, saj vemo, da za matriko $V$ obstaja inverz. Tako dobimo
$$
V^{-1} S_1 x_i = \lambda_i V^T x_i
\text{.}
$$
Enačbo lahko dodatno razčlenimo
$$
V^{-1} S_1 V^{-T} V^T x_i = \lambda_i V^T x_i
\text{.}
$$
Ker je matrika $S_1$ simetrična, je tudi matrika $V^{-1} S_1 V^{-T}$ simetrična, saj velja
$$
(V^{-1} S_1 V^{-T})^T =  (V^{-T})^T S_1^T (V^{-1})^T =  V^{-1} S_1 V^{-T}
\text{.}
$$
Simetrično matriko pa lahko diagonaliziramo v bazi ortonormiranih lastnih vektorjev in tako dobimo takšno matriko $Y$, da velja $Y Y^T = Y^T Y = I $ in 
$$
V^{-1} S_1 V^{-T} = Y \Lambda Y^T
\text{,}
$$
kjer je $\Lambda = \text{diag}\left(\lambda_1, \ldots, \lambda_m \right)$ diagonalna matrika. S preoblikovanjem zgornje enačbe pa dobimo
$$
S_1 = VY \Lambda Y^TV^T = X^{-T} \Lambda X^{-1}
\text{,}
$$
kjer smo dodatno definirali matriko $X := V^{-T} Y^{-T}$.
Poleg matrike $S_1$ pa lahko preoblikujemo tudi matriko $S_2$,
$$
S_2 = VV^T = VYY^TV^T = X^{-T}X^{-1}
\text{.}
$$
Zgornji enačbi lahko tudi obrnemo, tako da je $X^T S_1 X = \Lambda$ in $X^T S_2 X = I_m$.

Iz posplošenega problema lastnih vrednosti $\textit{\eqref{pos-prob-last-v}}$ lahko vidimo, da sta $\lambda_i$ in $x_i$ ravno lastna vrednost in pripadajoči lastni vektor za matriko $S_2^{-1} S_1$. Ker je matrika $S_1$ simetrično pozitivno semidefinitna, saj je sestavljena kot produkt matrike in transponiranke te matrike, so vse njene lastne vrednosti $\lambda_i \geq 0$ za $i = 1, \ldots, m$. Z uporabo permutacijskih matrik, lahko matriko $\Lambda$ preuredimo tako, da za  $q = \text{rang} \left( S_1 \right)$ velja $\lambda_1 \geq \ldots \geq \lambda_q > \lambda_{q+1} = \ldots = \lambda_m = 0$.

Optimizacijski kriterij ima tako sledečo obliko
\begin{equation} \label{pospJ1}
\begin{split}
J_1\left(G\right)
=&
\text{sled} \left( (G^T S_2 G)^{-1} G^T S_1 G \right)
%\\
\\
=&
\text{sled} \left( (G^T X^{-T}X^{-1} G)^{-1} G^T X^{-T} \Lambda X^{-1} G \right)
%\\
\\
=&
\text{sled} \left( G^{-1} X X^{T} G^{-T} G^T X^{-T} \Lambda X^{-1} G \right)
%\\
\\
=&
\text{sled} \left( \tilde{G}^{-1} (\tilde{G}^T)^{-1} \tilde{G}^T \Lambda \tilde{G} \right)
%\\
\\
= &
\text{sled} \left( (\tilde{G}^T \tilde{G})^{-1} \tilde{G}^T \Lambda \tilde{G} \right)
\text{,} 
\end{split}
\end{equation}
kjer je matrika $\tilde{G} := X^{-1} G \in \mathbb{R}^{m \times \ell}$. Ker vemo, da ima matrika $X$ poln rang, ima matrika $\tilde{G}$ rang enak številu stolpcev (torej $\ell$) in tako lahko na njej naredimo $QR$ razcep in tako dobimo matriko $Q \in \mathbb{R}^{m \times \ell}$, ki ima ortonormirane stolpce in nesingularno matriko $R \in \mathbb{R}^{\ell \times \ell}$, da velja $\tilde{G} = QR$. Tako lahko zgornjo enačbo $\textit{\eqref{pospJ1}}$ dodatno preoblikujemo
\begin{equation}
\begin{split}
J_1(G)
=&
\text{sled} \left( ((QR)^T QR)^{-1} (QR)^T \Lambda QR \right)
\\
=&
\text{sled} \left( (R^T Q^T QR)^{-1} R^T Q^T \Lambda QR \right)
\\
=&
\text{sled} \left( (R^T R)^{-1} R^T Q^T \Lambda QR \right)
\nonumber \\
=&
\text{sled} \left( R^{-1} Q^T \Lambda QR \right)
\text{.}
\end{split}
\end{equation}
Ker pa vemo, da imata podobni matriki enako sled velja
\begin{equation}
J_1(G)
=
\text{sled} \left( R^{-1} Q^T \Lambda QR \right)
\nonumber \\
=
\text{sled} \left(Q^T \Lambda Q \right)
\text{.}
\end{equation}
Tako lahko vidimo, da je po diagonalizaciji matrik $S_1$ in $S_2$ maksimizacija optimizacijskega kriterija odvisna le še od matrike $Q$, ki predstavlja ortonormirano bazo za matriko $X^{-1} G$, torej
\begin{align*}
\max_G J_1(G) = \max_{Q^TQ=I} \text{sled} \left(Q^T \Lambda Q \right)
\text{.}
\end{align*}
Ker pa vemo, da je sled matrike enaka vsoti lastnih vrednosti in da ima matrika $Q$ ortonormirane stolpce, velja
\begin{align*}
\max_{Q^TQ=I} \text{sled} \left(Q^T \Lambda Q \right)
\leq
\lambda_1 + \cdots + \lambda_q
=
\text{sled} \left( S_2^{-1} S_1 \right)
\text{.}
\end{align*}
Za vsak $\ell \geq q$ optimizacijski kriterij doseže svoj maksimum pri izbiri
$$ Q = 
\begin{bmatrix}
I_{\ell} \\
0_{m-\ell, \ell}
\end{bmatrix}
\hspace{4mm}
\text{oziroma za}
\hspace{4mm}
G = 
X
\begin{bmatrix}
I_{\ell} \\
0_{m-\ell, \ell}
\end{bmatrix}
R
\text{.}
$$
Preslikava $G$, za katero maksimizacijski kriterij doseže svoj maksimum, pa ni enolična, saj za katerokoli nesingularno matriko $W \in \mathbb{R}^{\ell \times \ell}$ velja
\begin{equation}
\begin{split}
J_1(GW) =& 
\text{sled}\left( (W^T G^T S_2 G W)^{-1} W^T G^T S_1 GW \right)
\\
=&
\text{sled}\left( W^{-1}(G^T S_2 G)^{-1}W^{-T} W^T G^T S_1 GW \right)
\\
=&
\text{sled}\left( W^{-1}(G^T S_2 G)^{-1}G^T S_1 GW \right)
\text{.}
\nonumber
\end{split}
\end{equation}
Ker pa velja, da imata podobni matriki enako sled, lahko zgornjo enačbo množimo iz leve z $W$ in iz desne z $W^{-1}$ in tako dobimo
\begin{align*}
J_1(GW)
=
J_1(G)
\text{.}
\end{align*}
V kolikor za nesingularno matriko $W$ izberemo $R^{-1}$ (ta obstaja, saj je $R$ po definiciji nesingularna), je maksimum optimizacijskega kriterija $J_1(G)$ dosežen tudi za 
$$ G = X
\begin{bmatrix}
I_{\ell} \\
0_{m-\ell, \ell}
\end{bmatrix}
\text{.}
$$
Tako smo ugotovili, da za $\ell \geq q = rang(S_1)$ velja
$$
\text{sled} \left( (G^T S_2 G)^{-1} G^T S_1 G \right)
=
\text{sled} \left( S_2^{-1} S_1 \right)
\text{,}
$$
v kolikor preslikavo $G \in \mathbb{R}^{m \times \ell}$
sestavimo iz $\ell$ lastnih vektorjev matrike $S_2^{-1} S_1$, ki pripadajo $\ell$ največjim lastnim vrednostim te matrike.

Po ugotovitvi iz prvega poglavja, je matrika $S_2$ lahko nesingularna le, ko velja $m \leq n$, oziroma, ko je število pridobljenih podatkov večje od dimenzije posameznega podatka. V nasprotnem primeru trenutne rešitve za maksimizacijski kriterij ne moremo uporabiti. Za nadaljevanje zapišimo $\lambda_i$ iz enačbe $\textit{\eqref{pos-prob-last-v}}$ kot $\alpha_i^2/\beta_i^2$ in tako se naš problem posploši na
\begin{equation} \label{alpha-beta}
\beta_i^2 S_1 x_i = \alpha_i^2 S_2 x_i
\text{.}
\end{equation}
V naslednjem poglavju maksimizacijo optimizacijskega kriterija $J_1(G)$ posplošimo tudi na primer, ko je matrika $S_2$ singularna.

\subsection{Posplošitev optimizacijskega kriterija za singularno matriko $S_2$}

V nadaljevanju obravnavajmo maksimizacijo prejšnjega optimizacijskega kriterija $J_1$, kjer matrik $S_1$ in $S_2$ ne izbiramo več, temveč te določimo kot $S_1 = S_B$ in $S_2 = S_W$. Iščemo preslikavo $G$, ki zadošča pogojema
\begin{equation} \label{pogoji-G}
\min_G \text{sled} \left( G^T S_W G \right)
\hspace{4mm}
\text{in}
\hspace{4mm}
\max_G \text{sled} \left( G^T S_B G \right)
\text{.}
\end{equation}
Za iskanje vektorjev $x_i$ iz enačbe \textit{\eqref{alpha-beta}} uporabimo posplošeni singularni razcep. In sicer izračunamo posplošen singularen razcep, podan v \textsl{izreku \textbf{2}} in enačbah \textit{\eqref{def-X}}, \textit{\eqref{Van-Loan-Ka}} in \textit{\eqref{Van-Loan-Kb}} na matriki 
$K := 
\begin{bmatrix}
H_B^T \\
H_W^T
\end{bmatrix}
\in \mathbb{R}^{2n \times m}$.
Tako dobimo ortogonalne matrike $U \in \mathbb{R}^{n \times n}$, $V \in \mathbb{R}^{n \times n}$ in $X \in \mathbb{R}^{m \times m}$ ter matriki $\Sigma_A$ in $\Sigma_B$, da velja
\begin{equation} \label{exp-H_b}
H_B^T = U 
\begin{bmatrix}
\Sigma_A, \hspace{2mm} 0 
\end{bmatrix} X^{-1}
\end{equation}
in
\begin{equation} \label{exp-H_w}
H_W^T = V 
\begin{bmatrix}
\Sigma_B, \hspace{2mm} 0 
\end{bmatrix} X^{-1}
\text{,}
\end{equation}
kjer sta matriki $\Sigma_A$ in $\Sigma_B$ sledečih oblik
$$
\Sigma_A = 
\begin{bmatrix} 
I_r &  & \\
 & D_A & \\
 & & 0_{n-r-s, k-r-s}  
\end{bmatrix}
\hspace{1mm}
\text{,}
\hspace{4mm}
\Sigma_B =
\begin{bmatrix}
0_{n-k+r, r} & & \\ 
 & D_B & \\ 
 & & I_{k-r-s}
\end{bmatrix} \text{,}
$$ matriki $D_A$ in $D_B$ pa sta oblike
$$
D_A =
\begin{bmatrix}
\alpha_{r+1} & & \\
 & \ddots & \\
 & & \alpha_{r+s}
\end{bmatrix}
\hspace{1mm}
\text{,}
\hspace{4mm}
D_B = 
\begin{bmatrix}
\beta_{r+1} & & \\
 & \ddots & \\
 & & \beta_{r+s}
\end{bmatrix}
\text{,}
$$ kjer je
$r = \text{rang}(K) - \text{rang}(H_W^T)$,  $s = \text{rang}(H_B^T) + \text{rang}(H_W^T) - \text{rang}(K)$ ter $k = \text{rang}(K)$. Za elemente matrik $D_A$ in $D_B$ pa velja
$1 > \alpha_{r+1} \geq \ldots \geq \alpha_{r+s} > 0$ in $ 0 < \beta_{r+1} \leq \ldots \leq \beta_{r+s} < 1$.
Sedaj si podrobneje oglejmo matriki $S_B = H_B H_B^T$ in $S_W = H_W H_W^T$.
Za produkt matrik $H_B H_B^T$ po \textit{\eqref{exp-H_b}} velja
\begin{equation} %\label{H_BH_B^T}
\begin{split}
\nonumber
H_B H_B^T =&
X^{-T}
\begin{bmatrix}
\Sigma_A^T \\
0^T
\end{bmatrix}
U^T
U \left[\Sigma_A, \hspace{2mm} 0 \right] X^{-1}
\\
=& 
X^{-T}
\begin{bmatrix}
\Sigma_A^T \Sigma_A & 0 \\
0 & 0
\end{bmatrix}
X^{-1} \text{,}
\end{split}
\end{equation}
kjer je matrika $\Sigma_A^T \Sigma_A \in \mathbb{R}^{k \times k}$ enaka diagonalni matriki $\text{diag}\left(\alpha_1^2, \ldots, \alpha_k^2 \right)$. V kolikor tu gledamo le $i$-ti stolpec obrnljive matrike $X$, ki ga označimo z $x_i$, ter dodatno definiramo $\alpha_i = 0$ za $i = k+1, \ldots, m$, dobimo
\begin{equation} \label{H_B}
x_i^T H_B H_B^T x_i = \alpha_i^2
\hspace{1mm}
\text{,}
\hspace{4mm}
i = 1, 2, \ldots, m
\text{.}
\end{equation}
Za produkt matrik $H_W H_W^T$ po podobnem izračunu kot za $H_B H_B^T$ velja
\begin{equation} %\label{H_WH_W^T}
H_W H_W^T =
X^{-T}
\begin{bmatrix}
\Sigma_B^T \Sigma_B & 0 \\
0 & 0
\end{bmatrix}
X^{-1} \text{.}
\nonumber
\end{equation}
Prav tako velja tudi 
\begin{equation} \label{H_W}
x_i^T H_W H_W^T x_i = \beta_i^2
\hspace{1mm}
\text{,}
\hspace{4mm}
i = 1, 2, \ldots, m
\text{,}
\end{equation}
kjer smo dodatno definirali $\beta_i = 0$ za $i = k+1, \ldots, m$.

%\textbf{-- DOPOLNI!}

Če združimo enačbi \textit{\eqref{H_B}} in \textit{\eqref{H_W}}, dobimo enačbo
$$
\frac{1}{\beta_i^2} x_i^T H_W H_W^T x_i = \frac{1}{\alpha_i^2} x_i^T H_B H_B^T x_i
\text{,}
$$
ki jo z leve množimo z inverznim vektorjem vektorja $x_i^T$ (ta obstaja, saj je matrika $X$, in posledično tudi matrika $X^T$, obrnljiva) in tako dobimo
\begin{equation} \label{alpha-beta-1}
\alpha_i^2 H_W H_W^T x_i = \beta_i^2 H_B H_B^T x_i
\text{.}
\end{equation}
Tako dobimo enak problem kot v \textit{\eqref{alpha-beta}}.
Najprej bomo potegnili vzporednice med to metodo in metodo iz prejšnjega poglavja, ki velja le za nesingularno matriko $S_W$, nato pa si bomo pogledali primer rešitve tega problema za singularno matriko $S_W$.

\subsubsection{Matrika $H_W$ polnega ranga}
V tem primeru bo zagotovo veljalo $n > m$. Matrika $H_W^T$ bo v tem primeru imela poln stolpični rang, torej $\text{rang}(H_W^T) = m$, iz česar pa sledi $r = m - m = 0$, $s = \text{rang}(H_B^T) + m - m = \text{rang}(H_B^T)$ ter $k = m$. Od tod dobimo, da je $\beta_i \neq 0 \hspace{2mm}  \text{za} \hspace{2mm}\forall \hspace{1mm} i = 1, \ldots, m \text{.}$ Posledično lahko enačbo \textit{\eqref{alpha-beta-1}} delimo z $\beta_i$ in dobimo 
$$
\frac{\alpha_i^2}{\beta_i^2} H_W H_W^T x_i =  H_B H_B^T x_i
\text{.}
$$
Ker pa nam posplošeni singularni razcep vrne pare singularnih vrednosti $(\alpha_i, \beta_i)$ urejene v sledečem vrstnem redu $1 > \alpha_{r+1} \geq \ldots \geq \alpha_{r+s} > 0$ in $0 < \beta_{r+1} \leq \ldots \leq \beta_{r+s} < 1$, velja, da so koeficienti $\lambda_i = \frac{\alpha_i^2}{\beta_i^2}$ v padajočem vrstnem redu, saj za poljuben $i \in {1, \ldots, s}$ velja
$$
\frac{\alpha_{r+i}}{\beta_{r+i}} = 
\frac{\alpha_{r+i}}{\beta_{r+i+1}} \frac{\beta_{r+i + 1}}{\beta_{r+i}} 
\geq 
\frac{\alpha_{r+i}}{\beta_{r+i+1}}
\geq
\frac{\alpha_{r+i+1}}{\beta_{r+i+1}}
\text{.}
$$
%so koeficienti $\frac{\alpha_i}{\beta_i}$ nenaraščajoči, 
Dodatno iz $\alpha_i \geq 0$ in $\beta_i \geq 0$ za $i = 1, \ldots, m$ sledi, da so tudi koeficienti $\frac{\alpha_i^2}{\beta_i^2}$ v padajočem vrstnem redu. Sledi, da za optimalno preslikavo $G$ potrebujemo le prvih $k - 1$ stolpcev matrike $X$, saj v tem primeru velja kar $k = m$, mi pa iščemo preslikavo, ki slika v prostor dimenzije manjše od $m$.

%\textbf{-- Bi bilo morda tu smiselno vzeti kar vseh k stolpcev?} -- potem tu ne bi spremenil dimenzije

\subsubsection{Matrika $H_W$ nepolnega ranga}
Do primera, ko je matrika $H_W$ nepolnega ranga, pride vedno, ko velja $m>n$. Tu tako ne moremo definirati lastnih vektorjev matrike $S_W^{-1} S_B$ in tako klasična diskriminantna analiza odpove. Recimo, da imamo singularni vektor $x_i$, ki leži v jedru matrike $S_W$ (torej $x_i \in \ker (S_W)$).  Iz \textit{\eqref{alpha-beta}} vidimo, da potem velja ena izmed možnosti: ta vektor leži tudi v jedru matrike $S_B$ ali pa je pripadajoča singularna vrednost $\beta_i$ enaka 0. Ločimo torej dve možnosti:
\begin{enumerate}
\item{$x_i \in \ker (S_W) \cap \ker (S_B)$}

V tem primeru je enačbi \textit{\eqref{alpha-beta}} zadoščeno za poljubna $\alpha_i$ in $\beta_i$. To bo primer za $m-k$ desnih stolpcev preslikave $X$. Ti stolpci so očitno v jedru preslikave $S_W$, saj se po \textit{\eqref{eq:5}} preslikajo v 0. Premislimo, ali se nam te stolpce preslikave $X$ splača vključiti v preslikavo $G$. Velja
$$
\text{sled}(G^T S_B G) = \sum_{j = 1}^{\ell} g_j^T S_B g_j
\hspace{2mm}
\text{in}
\hspace{2mm}
\text{sled}(G^T S_W G) = \sum_{j = 1}^{\ell} g_j^T S_W g_j
\text{,}
$$
kjer $g_j$ predstavlja $j$-ti stolpec preslikave $G$. Ker velja $x_i^T S_W x_i = 0$ in $x_i^T S_B x_i = 0$, dodajanje teh stolpcev v preslikavo $G$ ne bo imelo vpliva niti na maksimizacijo $\text{sled} \left( G^T S_W G \right)$, niti na minimizacijo $\text{sled} \left( G^T S_B G \right)$ iz \textit{\eqref{pogoji-G}}. Posledično teh stolpcev $x_i$ ne vključimo v $G$.

\item{$x_i \in \ker (S_W) - \ker (S_B) 	\Rightarrow \beta_i = 0$}

Iz $\beta_i = 0$ in \textit{\eqref{alpha+beta-GSVD}} sledi, da je $\alpha_i = 1$, iz česar sledi $\lambda_i = \infty$. Ti vektorji $x_i$ bodo predstavljali najbolj leve stolpce matrike $X$. Če te stolpce vključimo v preslikavo $G$, bomo na pogoje iz \textit{\eqref{pogoji-G}} vplivali tako, da bomo $\text{sled}(G^T S_B G)$ povečali, medtem, ko bo $\text{sled}(G^T S_W G)$ ostala nespremenjena. Tako lahko sklepemo, da te stolpce vključimo v $G$.
\end{enumerate}
Iz opisanega zaključimo, da tudi, ko je matrika $S_W$ singularna, preslikavo $G \in \mathbb{R}^{m \times \ell}$ sestavimo iz $\ell$ levih stolpcev matrike $X$. Od tod sledi naslednji algoritem.

\newpage 
\section{Algoritem}

\begin{algorithm}
\caption{Posplošena linearna diskriminantna analiza z uporabo posplošenega singularnega razcepa}
\hspace*{\algorithmicindent} \textbf{Vhod:} Matrika $A \in \mathbb{R}^{m \times n}$, katere podatki so $k$ razredih in vektor $a \in \mathbb{R}^{m \times 1}$, katerega razred je neznan.
\\
\hspace*{\algorithmicindent} \textbf{Izhod:} Preslikava $G  \in \mathbb{R}^{m \times \ell}$, ki ohranja strukturo razredov podatkov z maksimizacijo $J_1(G) = \text{sled}\left((G^T S_W G)^{-1} G^T S_B G \right)$ ter poračuna $\ell$-dimenzionalno preslikavo $y$ vektorja $a$.
\begin{algorithmic}[1]
    \State Iz matrike A poračunamo matriki $H_B$ in $H_W$ z uporabo formule
      \begin{equation} \label{H_B2}
	H_B = \left[\sqrt{n_1} (c^{(1)} - c), \hspace{1mm} \sqrt{n_2} (c^{(2)} - c),\ldots, \hspace{1mm} \sqrt{n_k} (c^{(k)} - c)\right]
      \end{equation}
      in formule \textit{\eqref{H_W1}}.
    \State Izračunamo matriki $P$ in $Q$ iz singularnega razcepa matrike $K := \begin{bmatrix} H_B^T \\ H_W^T\end{bmatrix}$,
      $$
	K =
        P
	\begin{bmatrix} 
	R & 0_{k, m-k} \\
	0_{n+p-k, k} & 0_{n+p-k, m-k} 
	\end{bmatrix}
	Q^T \text{.} 
      $$
    \State Označimo $k := \text{rang}(K)$.
    \State Matriko $P$ razdelimo 
      $$
	P = \left[P_1 \hspace{2mm} P_2 \right]
		=
	\begin{bmatrix} 
	P_{11} & P_{12} \\
	P_{21} & P_{22} 
	\end{bmatrix},
	$$
	kjer so podmatrike matrike $P$ pa so sledečih dimenzij: $P_{11} \in \mathbb{R}^{p \times k}$, $P_{12} \in 				\mathbb{R}^{p \times (p+n-k)}$, $P_{21} \in \mathbb{R}^{n \times k}$ in $P_{22} \in \mathbb{R}^{n \times (p+n-		k)}$.
    \State Izračunamo matriko $W$ iz singularnega razcepa matrike $P_{11}$,
      $$
	P_{11} =
        U
	\begin{bmatrix} 
	\Sigma_A & 0 
	\end{bmatrix}
	W^T \text{.} 
      $$

    \State Za $G$ razglasimo prvih $\ell$ stolpcev matrike $X = Q
	\begin{bmatrix}
	R^{-1} W & 0 \\
	0 & I 
	\end{bmatrix}$.
    \State $y = G^T a$
  \end{algorithmic}
\end{algorithm}

\begin{opomba}
Algoritem v drugem koraku namesto matrike $H_B$ iz \textit{\eqref{H_B1}} uporabimo matriko $H_B := \left[\sqrt{n_1} (c^{(1)} - c), \hspace{1mm} \sqrt{n_2} (c^{(2)} - c),\ldots, \hspace{1mm} \sqrt{n_k} (c^{(k)} - c)\right]$, saj ta matrika na enak način določa razpršenost podatkov med posamezni razredi. Sled matrike, ki jo maksimiziramo, se tako le sorazmerno zmanjša, saj je razlika med centroidom posameznega razreda in centroida celotnih podatkov otežena s korenom števila podatkov v posameznem razredu. %Tu za utež razlike namesto bolj očitnega števila podatkov v posameznem razredu vzamemo koren števila podatkov v posameznem razredu, tako da so števila manjša, kar dodatno olajša računsko zahtevnost. 
To lahko naredimo, saj vemo, da je koren na intervalu $[0, \infty)$ injektivna preslikava. Matiko $H_B$ tu zamenjamo z namenom olajšanja računske zahtevnosti algoritma, saj je nova matrika dimenzije $m \times k$, kar je občutno manj, prav tako so tudi elementi matrike manjši.
\end{opomba}

\section{Uporaba algoritma}

\section{Priloge}

\begin{izrek}[Singularni razcep]
\label{izrek:SVD} Za vsako matriko $A \in \mathbb{R}^{m \times n}$, z lastnostjo $m \geq n$, obstaja singularni razcep 
$$A = U \Sigma V^T \text{,}$$
kjer sta $U \in \mathbb{R}^{m \times m}$ in $V \in \mathbb{R}^{n \times n}$ ortogonalni matriki, $\Sigma \in \mathbb{R}^{m \times n}$ je oblike
$$
\Sigma = 
\begin{bmatrix} 
\sigma_1 &  & \\
 & \ddots & \\
 & & \sigma_n  \\
 & & 
\end{bmatrix}$$
in $\sigma_1 \geq \sigma_2 \geq \ldots \geq \sigma_n \geq 0$ so singularne vrednosti matrike $A$.
\end{izrek}
\begin{proof}
Ker je $A^TA$ simetrična pozitivno semidefinitna matrika, so vse njene lastne vrednosti nenegativne. Označimo in uredimo jih kot
$$\sigma_1^2 \geq \sigma_2^2 \geq \ldots \geq \sigma_n^2 \geq 0 \text{.}$$
Ustrezni ortonormirani lastni vektorji $v_1, \ldots, v_n$ zadoščajo
$A^T A v_i = \sigma_i^2 v_i$ za $i = 1, \ldots, n \text{.}$
Naj bo $\sigma_r > 0$ in $\sigma_{r+1} = \cdots = \sigma_n = 0 \text{.}$
Matriko $V$ razdelimo na $V_1 = 
\left[ v_1, \ldots, v_r
\right]$ in $V_2 = 
\left[ v_{r+1}, \ldots, v_n
\right] \text{.}$ Iz
$$ (AV_2)^T (AV_2) = V_2^T A^T A V_2 = V_2^T \left[ 0, \ldots, 0 \right] = 0
$$
sledi $AV_2 = 0 \text{.}$ Sedaj definiramo $u_i := \frac{1}{\sigma_i} Av_i$ za $i = 1, \ldots, r \text{.}$
Vekorji $u_1, \ldots, u_r$ so ortonormirani, saj je
$$ u_i^T u_j = \frac{1}{\sigma_i \sigma_j} v_i^T A^T A v_j = \frac{\sigma_j}{\sigma_i} v_i^T v_j = \delta_{ij} \text{,} \hspace{3mm} i,j = 1, \ldots, r \text{,}
$$
kjer smo v zapisu uporabili t.i. \emph{Kroneckerjev delta}, definiran z $\delta_{ij} = 1$ za $i = j$ in $\delta_{ij} = 0$ za $i \neq j$. Označimo 
$U_1 = 
\left[ u_1 \hspace{2mm} \cdots \hspace{2mm} u_r
\right]$ in dopolnimo z $U_2 = 
\left[ u_{r+1} \hspace{2mm} \cdots \hspace{2mm} u_n
\right]$, da je $ U = \left[ U_1 \hspace{2mm} U_2 \right]$ ortogonalna matrika. Matrika $U^T A V$ ima obliko 
$$
U^T A V = 
\begin{bmatrix} 
U_1^T A V_1 & U_1^T A V_2 \\
U_2^T A V_1 & U_2^T A V_2
\end{bmatrix}
\text{.}
$$
Desna bloka sta zaradi $AV_2 = 0$ enaka $0$. Za $i = 1, \ldots, r$ in $k = 1, \ldots, m$ velja
$$ u_k^T A v_i = \sigma_i u_k^T u_i = \sigma_i \delta_{ik} \text{,}
$$
torej $U_2^T A V_1 = 0$ in $U_1^T A V_1 = diag(\sigma_1, \ldots, \sigma_r) \text{.}$ Dobimo singularni razcep $A = U \Sigma V^T \text{,}$ kjer je $S = diag(\sigma_1, \ldots, \sigma_r)$ in 
$$ \Sigma = 
\begin{bmatrix} 
S & 0 \\
0 & 0
\end{bmatrix}
\text{.}$$

\end{proof}
V primeru, ko velja $n>m$, dobimo singularni razcep za $A \in \mathbb{R}^{m \times n}$ tako, da transponiramo singularni razcep matrike $A^T$.

\begin{thebibliography}{99}

\bibitem{glavni-1}
P.~Howland in H.~Park, \emph{Generalizing Discriminant Analysis Using the Generalizing Singular Value Decomposition}, IEEE Trans.\ on Pet.\ Anakysis and Mach.\ Int.\ \textbf{25} (2004) 995--1006.

\bibitem{GSVD-2}
C.C.~Paige in M.A.~Sounders, \emph{Towards a Generalized Singular Value Decomposition}, SIAM J.\ Numerical Analysis \textbf{18} (1981) 398--405.

\bibitem{referenca-knjiga}
B.~Plestenjak, \emph{Razširjen uvod v numerične metode}, DMFA - založništvo, Ljubljana, 2015.

\bibitem{GSVD-1}
C.F.~Van Loan, \emph{Generalizing the Singular Value Decomposition}, SIAM J.\ Numerical Analysis \textbf{13} (1976) 76--83.

\bibitem{glavni-2}
J.~Ye, \emph{Characterization of a Family of Algorithms for Generalized Discriminant Analysis on Undersampled Problems}, J.\ of Mach.\ Lear.\ Res.\ \textbf{6} (2005) 483--502.

%% ZGLEDI --- VIRI
%\bibitem{referenca-clanek}
%I.~Priimek, \emph{Naslov "clanka}, okraj"sano ime revije \textbf{letnik revije} (leto izida) strani od--do.

%\bibitem{navodilaOMF}
%C.~Velkovrh, \emph{Nekaj navodil avtorjem za pripravo rokopisa}, Obzornik mat.\ fiz.\ \textbf{21} (1974) 62--64.

%\bibitem{vec-avtorjev}
%P.~Angelini, F.~Frati in M.~Kaufmann, \emph{Straight-line rectangular drawings of clustered graphs}, Discrete Comput.\ Geom.\ \textbf{45} (2011) 88--140.



%\bibitem{referenca-knjiga}
%I.~Priimek, \emph{Naslov knjige}, morebitni naslov zbirke  \textbf{zaporedna "stevilka}, zalo"zba, kraj, leto izdaje.

%\bibitem{glob}
%J.~Globevnik in M.~Brojan, \emph{Analiza I}, Matemati"cni rokopisi \textbf{25}, DMFA -- zalo"zni"stvo, Ljubljana, 2010.

%\bibitem{glob-vse}
%J.~Globevnik in M.~Brojan, \emph{Analiza I}, Matemati"cni rokopisi \textbf{25}, DMFA -- zalo"zni"stvo, Ljubljana, 2010; dostopno tudi na
%\url{http://www.fmf.uni-lj.si/~globevnik/skripta.pdf}.

%\bibitem{lang}
%S.~Lang, \emph{Fundamentals of differential geometry}, Graduate Texts in Mathematics {\bf 191}, Springer-Verlag, New York, 1999.



%\bibitem{referenca-clanek-v-zborniku}
%I.~Priimek, \emph{Naslov "clanka}, v: naslov zbornika (ur.\ ime urednika), morebitni naslov zbirke  \textbf{zaporedna "stevilka}, zalo"zba, kraj, leto izdaje, str.\ od--do.

%\bibitem{zbornik}
%S.~Cappell in J.~Shaneson, \emph{An introduction to embeddings, immersions and singularities in codimension two}, v: Algebraic and geometric topology, Part 2 (ur.\ R.~Milgram), Proc.\ Sympos.\ Pure Math.\ \textbf{XXXII}, Amer.\ Math.\ Soc., Providence, 1978, str.\ 129--149.



%\bibitem{diploma-magisterij}
%I.~Priimek, \emph{Naslov dela}, diplomsko/magistrsko delo, ime fakultete, ime univerze, leto.

%\bibitem{kalisnik}
%J.~Kali"snik, \emph{Upodobitev orbiterosti}, diplomsko delo, Fakulteta za matematiko in fiziko, Univerza v Ljubljani, 2004.



%\bibitem{referenca-spletni-vir}
%I.~Priimek, \emph{Naslov spletnega vira}, v: ime morebitne zbirke/zbornika, ki vsebuje vir, verzija "stevilka/datum, [ogled datum], dostopno na \url{spletni.naslov}.

%\bibitem{glob-splet}
%J.~Globevnik in M.~Brojan, \emph{Analiza 1}, verzija 15.~9.~2010, [ogled 12.~5.~2011], dostopno na \url{http://www.fmf.uni-lj.si/~globevnik/skripta.pdf}.

%\bibitem{wiki}
%\emph{Matrix (mathematics)}, v: Wikipedia: The Free Encyclopedia, [ogled 12.~5.~2011], dostopno na \url{http://en.wikipedia.org/wiki/Matrix_(mathematics)}.




\end{thebibliography}


\end{document}